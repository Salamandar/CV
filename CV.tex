\documentclass[11pt,a4paper]{moderncv}
\moderncvtheme[blue]{classic}
\usepackage[utf8]{inputenc}
\usepackage[top=2cm, bottom=1.1cm, left=2cm, right=2cm]{geometry}
\let\fax\undefined
\usepackage{marvosym}
% Largeur de la colonne pour les dates
\setlength{\hintscolumnwidth}{3.5cm}

\usepackage{colortbl,calc}
\newlength{\myseparatorlinewidth}
\newlength{\myseparatorspaceswidth}
\newcommand*{\cvitemsep}[3][.25em]{%
  \begin{tabular}{@{}p{\hintscolumnwidth}
                @{\hspace{\myseparatorspaceswidth}}
                !{\color{color1}\vrule width \myseparatorlinewidth}
                @{\hspace{\myseparatorspaceswidth}}
                p{\maincolumnwidth}@{}}%
    \raggedleft\hintstyle{#2} &\par{#3}%
  \end{tabular}%
  \par\addvspace{#1}}


\photo[64pt][0.pt]{photo.pdf}
\firstname{Félix}
\familyname{Piédallu}
\title{Ingénieur R\&D en Physique \& Nanosciences}
\address{15 rue Buffon}{38000 Grenoble}
\email{felix@piedallu.me}
\homepage{geekolloc.fr}
\mobile{06 51 41 32 48}
\extrainfo{20 ans}

\begin{document}
\setlength{\myseparatorlinewidth}{1pt}
\setlength{\myseparatorspaceswidth}{.5\separatorcolumnwidth-.5\myseparatorlinewidth}


\maketitle
%\cventry{Objet :}{Stage opérateur de Première Année}{}{}{}{}
\vspace*{-1cm}
\section{Formation}

\cventry{2013 -- 2016}{Diplôme d'Ingénieur}{Phelma}{Grenoble INP}{Physique \& Nanosciences}{Spécialité Optique-Microélectronique}
\cventry{2011 -- 2013}{Classe Préparatoire}{Lycée Charlemagne}{Paris}{Maths-Physique (MP*)}{}
\cventry{2011}{Baccalauréat Scientifique}{Lycée Claude Monet}{Paris}{Mention Bien}{}


\section{Réalisations techniques}
\cventry{2016}{Stage de fin d'études (5 mois)}{Institut Néel}{Grenoble}{}{Caractérisation de pointes fibrées pour le développement d'une pince optique plasmonique}
\cventry{2015}{Stage d'Application (10 semaines)}{Laboratoire Pierre Aigrain}{}{Paris}{Mise en place d'une expérience de transport quantique à très basse température}
\cventry{2014 -- 2015}{Président du Club Robotronik de Phelma}{}{}{}{}
\cventry{2014}{Stage Opérateur (8 semaines)}{}{à la DSI du Courier de La Poste}{}{Intégration sous Android d'un service de reconnaissance d'images sur serveur distant.}
\cventry{2013 -- 2014}{Participation à la Coupe de France de Robotique}{}{}{}{}


\section{Connaissances en informatique}
\cvitem{Langages maîtrisés}{C, C++, Gtk+, \LaTeX, Java (Android)}
\cvitem{Autres langages}{HTML, Python, Matlab, Rust}
\cvitem{Systèmes}{Linux (Arch, Debian \& Ubuntu), Raspberry Pi}


\section{Compétences Linguistiques}
\cvlanguage{Allemand}{Lu, écrit, parlé}{Première langue entre 2004 et 2013, séjours en Allemagne}
\cvlanguage{Anglais}{Lu, écrit, parlé couramment}{93\% au BULATS, C2}
\cvlanguage{Chinois Mandarin}{Lu, écrit, parlé}{10 ans d'apprentissage et un séjour en Chine}
\cvlanguage{Russe}{Bases acquises}{}


\section{Centres d'intérêt}
\cvitemsep{Informatique \\ Électronique \\ Domotique}{Participe à la Coupe de France de Robotique 
\newline Projets personnels d'informatique et d'électronique (\href{https://github.com/Salamandar}{\ComputerMouse \texttt{ Profil Github}})
\newline Participe à la rédaction du Wiki Ubuntu francophone}
\cvitem{Musique}{Pratique l'Accordéon depuis 9 ans, écoute du Jazz au Metal}
\cvitemsep{Sport}{Natation, Plongée
\newline Tir à l'arc}

\cvitem{Modélisme}{Aéromodélisme et Robotique}

\end{document}
