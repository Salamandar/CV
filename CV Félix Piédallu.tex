\documentclass[11pt,a4paper]{moderncv}
\moderncvtheme[blue]{classic}
\usepackage[utf8]{inputenc}
\usepackage[top=2cm, bottom=1.1cm, left=2cm, right=2cm]{geometry}
\let\fax\undefined
\usepackage{marvosym}
% Largeur de la colonne pour les dates
\setlength{\hintscolumnwidth}{3.5cm}

\usepackage{colortbl,calc}
\newlength{\myseparatorlinewidth}
\newlength{\myseparatorspaceswidth}
\newcommand*{\cvitemsep}[3][.25em]{%
  \begin{tabular}{@{}p{\hintscolumnwidth}
                @{\hspace{\myseparatorspaceswidth}}
                !{\color{color1}\vrule width \myseparatorlinewidth}
                @{\hspace{\myseparatorspaceswidth}}
                p{\maincolumnwidth}@{}}%
    \raggedleft\hintstyle{#2} &\par{#3}%
  \end{tabular}%
  \par\addvspace{#1}}


\photo[64pt][0.pt]{photo.pdf}
\firstname{Félix}
\familyname{Piédallu}
\title{Ingénieur R\&D en Physique \& Nanosciences}
\address{9 rue Barginet}{38000 Grenoble}
\email{felix@piedallu.me}
\homepage{geekolloc.fr}
\mobile{06 51 41 32 48}
\extrainfo{22 ans}

\begin{document}
\setlength{\myseparatorlinewidth}{1pt}
\setlength{\myseparatorspaceswidth}{.5\separatorcolumnwidth-.5\myseparatorlinewidth}


\maketitle
%\cventry{Objet :}{Stage opérateur de Première Année}{}{}{}{}
\vspace*{-1cm}

\section{Réalisations techniques}
\cventry{\textbf{2017 -- 2018}\\18 mois}{Ingénieur développement logiciel \texttt{C++}}{Keepixo}{Meylan}{}{Conception et développement d’un serveur de diffusion vidéo
\begin{itemize}
    \item Mise en place et gestion d’une API web
    \item Modernisation de la base de code, portage sous Linux et CMake
\end{itemize}}
\cventry{\textbf{2016}\\5 mois}{Stage de fin d'études}{Institut Néel}{Grenoble}{}{Caractérisation de pointes fibrées pour une pince optique plasmonique
\begin{itemize}
    \item Instrumentation et mesures optiques et photoniques
    \item Étude et caractérisation de faisceaux gaussiens et de Bessel
\end{itemize}}
\cventry{\textbf{2015}\\10 semaines}{Stage d'Application}{Laboratoire Pierre Aigrain}{}{Paris}{Mise en place d’une expérience de transport quantique à très basse température
\begin{itemize}
    \item Instrumentation très basse température : Cryostat à dilution
    \item Hautes fréquences : Fabrication et filtrage de lignes RF
\end{itemize}}
\cventry{\textbf{2014 -- 2015}}{Président du Club Robotronik de Phelma}{}{}{}{}
\cventry{\textbf{2014}\\8 semaines}{Stage Opérateur}{}{à la DSI du Courier de La Poste}{}{Intégration sous Android d'un service de reconnaissance d'images sur serveur distant.}
\cventry{\textbf{2013 -- 2018}}{Participation à la Coupe de France de Robotique}{}{}{}{}


\section{Formation}
\cventry{\textbf{2013 -- 2016}}{Diplôme d'Ingénieur}{Phelma}{Grenoble INP}{Physique \& Nanosciences}{Spécialité Optique-Microélectronique}
\cventry{\textbf{2011 -- 2013}}{Classe Préparatoire}{Lycée Charlemagne}{Paris}{Maths-Physique (MP*)}{}
\cventry{\textbf{2011}}{Baccalauréat Scientifique}{Lycée Claude Monet}{Paris}{Mention Bien}{}


\section{Connaissances en informatique}
\cvitem{Langages maîtrisés}{C, C++, Gtk+, \LaTeX}
\cvitem{Autres langages}{HTML, Python, Matlab, Rust, Java (Android)}
\cvitem{Systèmes}{Linux (Arch, Debian \& Ubuntu), Raspberry Pi, Embarqué}


\section{Compétences Linguistiques}
\cvlanguage{Anglais}{Lu, écrit, parlé couramment}{93\% au BULATS, C2}
\cvlanguage{Allemand}{Lu, écrit, parlé}{Première langue entre 2004 et 2013, séjours en Allemagne}
\cvlanguage{Chinois Mandarin}{Lu, écrit, parlé}{10 ans d'apprentissage et un séjour en Chine}
%\cvlanguage{Russe}{Bases acquises}{}


\section{Centres d'intérêt}
\cvitemsep{Informatique \\ Électronique \\ Domotique}{
          Participe à la Coupe de France de Robotique (\href{https://phoenixrobotik.tech}{\ComputerMouse \texttt{Phoenix Robotik}})
\newline  Projets personnels d'informatique et d'électronique (\href{https://github.com/Salamandar}{\ComputerMouse \texttt{Profil Github}})
%\newline  Participe à la rédaction du Wiki Ubuntu francophone
}
\cvitem{Musique}{Écoute du Jazz au Metal}
\cvitem{Sport}{Natation, Plongée, Tir à l'arc}

\end{document}
