\documentclass[11pt,a4paper]{moderncv}
\moderncvtheme[blue]{classic}
\usepackage[utf8]{inputenc}
\usepackage[top=2cm, bottom=1.1cm, left=2cm, right=2cm]{geometry}
\let\fax\undefined
\usepackage{marvosym}
% Largeur de la colonne pour les dates
\setlength{\hintscolumnwidth}{3.5cm}

\usepackage{colortbl,calc}
\newlength{\myseparatorlinewidth}
\newlength{\myseparatorspaceswidth}
\newcommand*{\cvitemsep}[3][.25em]{
  \begin{tabular}{@{}p{\hintscolumnwidth}
      @{\hspace{\myseparatorspaceswidth}}
      !{\color{color1}\vrule width \myseparatorlinewidth}
      @{\hspace{\myseparatorspaceswidth}}
      p{\maincolumnwidth}@{}
    }
    \raggedleft\hintstyle{#2} &\par{#3}
  \end{tabular}
  \par\addvspace{#1}
}



\photo[64pt][0.pt]{photo.pdf}
\firstname{Félix}
\familyname{Piédallu}
\title{Ingénieur logiciel et électronique}
\address{9 rue Barginet}{38000 Grenoble}
\email{felix@piedallu.me}
\homepage{salamandar.fr}
\mobile{\href{tel:+33 6 51 41 32 48}{06 51 41 32 48}}
\extrainfo{24 ans}

\begin{document}
\setlength{\myseparatorlinewidth}{1pt}
\setlength{\myseparatorspaceswidth}{.5\separatorcolumnwidth-.5\myseparatorlinewidth}


\maketitle
\vspace*{-1cm}

\section{Réalisations techniques}

\cventry{\textbf{2020 -- 2022}\\2.5 ans -- CDI}{Développeur logiciel et électronique}{Elsys Design+Kalray}{Grenoble}{}{
  \begin{itemize}
    \item Développement logiciel sur une architecture spécifique
    \item Modernisation et réécriture de la base de tests automatisé
    \item Évaluation et design d’un \textit{shield} pour capteur de distance "Time Of Flight"
  \end{itemize}
}
\cventry{\textbf{2018 -- 2019}\\1.5 ans -- CDD}{Ingénieur Fablab}{Grenoble INP - Phelma}{Grenoble}{}{
  \begin{itemize}
    \item Gestion d'un atelier avec imprimantes 3D et CNC
    \item Développement de travaux pratiques étudiants
  \end{itemize}
}
\cventry{\textbf{2017 -- 2018}\\1.5 ans -- CDI}{Ingénieur développement logiciel \texttt{C++}}{Keepixo}{Meylan}{}{
  Conception et développement d'un serveur de diffusion vidéo
  \begin{itemize}
    \item Modernisation de la base de code, portage sous Linux et CMake
  \end{itemize}
}
\cventry{\textbf{2016}\\5 mois}{Stage de fin d'études}{Institut Néel}{Grenoble}{}{
  Caractérisation de pointes fibrées pour une pince optique plasmonique
  \begin{itemize}
    \item Instrumentation et mesures optiques et photoniques
  \end{itemize}
}
\cventry{\textbf{2015}\\10 semaines}{Stage d'Application}{Laboratoire Pierre Aigrain}{}{Paris}{
  Mise en place d'une expérience de transport quantique à très basse température
  \begin{itemize}
    \item Hautes fréquences : Fabrication et filtrage de lignes RF
  \end{itemize}
}
\cventry{\textbf{2014}\\8 semaines}{Stage Opérateur}{}{à la DSI du Courier de La Poste}{}{
  Intégration sous Android d'un service de reconnaissance d'images sur serveur distant.
}
\cventry{\textbf{2014 -- 2015}}{Président du Club Robotronik de Phelma}{}{}{}{}
\cventry{\textbf{2013 -- 2020}}{Participation à la Coupe de France de Robotique}{}{}{}{
  Avec Robotronik Phelma, puis Phenix Robotik et en tant que bénévole.
}


\section{Formation}
\cventry{\textbf{2013 -- 2016}}{Diplôme d'Ingénieur}{}{Grenoble INP - Phelma}{\textit{Physique \& Nanosciences}}{Spécialité Optique-Microélectronique}
\cventry{\textbf{2011 -- 2013}}{Classe Préparatoire}{Lycée Charlemagne}{Paris}{Maths-Physique (MP*)}{}
\cventry{\textbf{2011}}{Baccalauréat Scientifique}{Lycée Claude Monet}{Paris}{Mention Bien}{}


\section{Connaissances technologiques}
\cvitemsep{Technologies maîtrisées}{
          C, C++, Python, Meson, Yocto
\newline  Arduino/STM32, Bare-Metal/RTOS
\newline  Design électronique (KiCAD)
\newline  CAO (OpenSCAD, FreeCAD), Impression 3D
}
\cvitem{Autres}{Rust, Java (Android), html/css, Zephyr}
\cvitem{Systèmes}{Linux (Archlinux, Debian), Raspberry Pi, Embarqué}


\section{Compétences Linguistiques}
\cvlanguage{Anglais}{Lu, écrit, parlé couramment}{93\% au BULATS, C2}
\cvlanguage{Allemand}{Lu, écrit, parlé}{Première langue entre 2004 et 2013, séjours en Allemagne}
\cvlanguage{Chinois Mandarin}{Lu, écrit, parlé}{10 ans d'apprentissage et un séjour en Chine}
%\cvlanguage{Russe}{Bases acquises}{}


\section{Centres d'intérêt}
\cvitemsep{Informatique \\ Électronique}{
          Participe à la Coupe de France de Robotique (\href{https://phenixrobotik.fr}{\ComputerMouse \texttt{Phenix Robotik}})
\newline  Projets personnels d'informatique et d'électronique (\href{https://github.com/Salamandar}{\ComputerMouse \texttt{Profil Github}})
%\newline  Participe à la rédaction du Wiki Ubuntu francophone
}
\cvitem{Musique}{Écoute du Jazz au Metal}
\cvitem{Sport}{Natation, Plongée, Tir à l'arc}

\end{document}
